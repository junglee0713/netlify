\documentclass[]{article}
\usepackage{lmodern}
\usepackage{amssymb,amsmath}
\usepackage{ifxetex,ifluatex}
\usepackage{fixltx2e} % provides \textsubscript
\ifnum 0\ifxetex 1\fi\ifluatex 1\fi=0 % if pdftex
  \usepackage[T1]{fontenc}
  \usepackage[utf8]{inputenc}
\else % if luatex or xelatex
  \ifxetex
    \usepackage{mathspec}
  \else
    \usepackage{fontspec}
  \fi
  \defaultfontfeatures{Ligatures=TeX,Scale=MatchLowercase}
\fi
% use upquote if available, for straight quotes in verbatim environments
\IfFileExists{upquote.sty}{\usepackage{upquote}}{}
% use microtype if available
\IfFileExists{microtype.sty}{%
\usepackage{microtype}
\UseMicrotypeSet[protrusion]{basicmath} % disable protrusion for tt fonts
}{}
\usepackage[margin=1in]{geometry}
\usepackage{hyperref}
\hypersetup{unicode=true,
            pdftitle={Homework 5},
            pdfborder={0 0 0},
            breaklinks=true}
\urlstyle{same}  % don't use monospace font for urls
\usepackage{color}
\usepackage{fancyvrb}
\newcommand{\VerbBar}{|}
\newcommand{\VERB}{\Verb[commandchars=\\\{\}]}
\DefineVerbatimEnvironment{Highlighting}{Verbatim}{commandchars=\\\{\}}
% Add ',fontsize=\small' for more characters per line
\usepackage{framed}
\definecolor{shadecolor}{RGB}{248,248,248}
\newenvironment{Shaded}{\begin{snugshade}}{\end{snugshade}}
\newcommand{\KeywordTok}[1]{\textcolor[rgb]{0.13,0.29,0.53}{\textbf{#1}}}
\newcommand{\DataTypeTok}[1]{\textcolor[rgb]{0.13,0.29,0.53}{#1}}
\newcommand{\DecValTok}[1]{\textcolor[rgb]{0.00,0.00,0.81}{#1}}
\newcommand{\BaseNTok}[1]{\textcolor[rgb]{0.00,0.00,0.81}{#1}}
\newcommand{\FloatTok}[1]{\textcolor[rgb]{0.00,0.00,0.81}{#1}}
\newcommand{\ConstantTok}[1]{\textcolor[rgb]{0.00,0.00,0.00}{#1}}
\newcommand{\CharTok}[1]{\textcolor[rgb]{0.31,0.60,0.02}{#1}}
\newcommand{\SpecialCharTok}[1]{\textcolor[rgb]{0.00,0.00,0.00}{#1}}
\newcommand{\StringTok}[1]{\textcolor[rgb]{0.31,0.60,0.02}{#1}}
\newcommand{\VerbatimStringTok}[1]{\textcolor[rgb]{0.31,0.60,0.02}{#1}}
\newcommand{\SpecialStringTok}[1]{\textcolor[rgb]{0.31,0.60,0.02}{#1}}
\newcommand{\ImportTok}[1]{#1}
\newcommand{\CommentTok}[1]{\textcolor[rgb]{0.56,0.35,0.01}{\textit{#1}}}
\newcommand{\DocumentationTok}[1]{\textcolor[rgb]{0.56,0.35,0.01}{\textbf{\textit{#1}}}}
\newcommand{\AnnotationTok}[1]{\textcolor[rgb]{0.56,0.35,0.01}{\textbf{\textit{#1}}}}
\newcommand{\CommentVarTok}[1]{\textcolor[rgb]{0.56,0.35,0.01}{\textbf{\textit{#1}}}}
\newcommand{\OtherTok}[1]{\textcolor[rgb]{0.56,0.35,0.01}{#1}}
\newcommand{\FunctionTok}[1]{\textcolor[rgb]{0.00,0.00,0.00}{#1}}
\newcommand{\VariableTok}[1]{\textcolor[rgb]{0.00,0.00,0.00}{#1}}
\newcommand{\ControlFlowTok}[1]{\textcolor[rgb]{0.13,0.29,0.53}{\textbf{#1}}}
\newcommand{\OperatorTok}[1]{\textcolor[rgb]{0.81,0.36,0.00}{\textbf{#1}}}
\newcommand{\BuiltInTok}[1]{#1}
\newcommand{\ExtensionTok}[1]{#1}
\newcommand{\PreprocessorTok}[1]{\textcolor[rgb]{0.56,0.35,0.01}{\textit{#1}}}
\newcommand{\AttributeTok}[1]{\textcolor[rgb]{0.77,0.63,0.00}{#1}}
\newcommand{\RegionMarkerTok}[1]{#1}
\newcommand{\InformationTok}[1]{\textcolor[rgb]{0.56,0.35,0.01}{\textbf{\textit{#1}}}}
\newcommand{\WarningTok}[1]{\textcolor[rgb]{0.56,0.35,0.01}{\textbf{\textit{#1}}}}
\newcommand{\AlertTok}[1]{\textcolor[rgb]{0.94,0.16,0.16}{#1}}
\newcommand{\ErrorTok}[1]{\textcolor[rgb]{0.64,0.00,0.00}{\textbf{#1}}}
\newcommand{\NormalTok}[1]{#1}
\usepackage{graphicx,grffile}
\makeatletter
\def\maxwidth{\ifdim\Gin@nat@width>\linewidth\linewidth\else\Gin@nat@width\fi}
\def\maxheight{\ifdim\Gin@nat@height>\textheight\textheight\else\Gin@nat@height\fi}
\makeatother
% Scale images if necessary, so that they will not overflow the page
% margins by default, and it is still possible to overwrite the defaults
% using explicit options in \includegraphics[width, height, ...]{}
\setkeys{Gin}{width=\maxwidth,height=\maxheight,keepaspectratio}
\IfFileExists{parskip.sty}{%
\usepackage{parskip}
}{% else
\setlength{\parindent}{0pt}
\setlength{\parskip}{6pt plus 2pt minus 1pt}
}
\setlength{\emergencystretch}{3em}  % prevent overfull lines
\providecommand{\tightlist}{%
  \setlength{\itemsep}{0pt}\setlength{\parskip}{0pt}}
\setcounter{secnumdepth}{0}
% Redefines (sub)paragraphs to behave more like sections
\ifx\paragraph\undefined\else
\let\oldparagraph\paragraph
\renewcommand{\paragraph}[1]{\oldparagraph{#1}\mbox{}}
\fi
\ifx\subparagraph\undefined\else
\let\oldsubparagraph\subparagraph
\renewcommand{\subparagraph}[1]{\oldsubparagraph{#1}\mbox{}}
\fi

%%% Use protect on footnotes to avoid problems with footnotes in titles
\let\rmarkdownfootnote\footnote%
\def\footnote{\protect\rmarkdownfootnote}

%%% Change title format to be more compact
\usepackage{titling}

% Create subtitle command for use in maketitle
\newcommand{\subtitle}[1]{
  \posttitle{
    \begin{center}\large#1\end{center}
    }
}

\setlength{\droptitle}{-2em}

  \title{Homework 5}
    \pretitle{\vspace{\droptitle}\centering\huge}
  \posttitle{\par}
    \author{}
    \preauthor{}\postauthor{}
    \date{}
    \predate{}\postdate{}
  

\begin{document}
\maketitle

\bigskip

(For questions \textbf{1, 2}) Consider the following paired data sets of
length \(20\):

\begin{Shaded}
\begin{Highlighting}[]
\NormalTok{x <-}\StringTok{ }\KeywordTok{c}\NormalTok{(}\FloatTok{6.82}\NormalTok{, }\FloatTok{1.44}\NormalTok{, }\FloatTok{9.39}\NormalTok{,  }\FloatTok{8.51}\NormalTok{, }\FloatTok{10.38}\NormalTok{,  }\FloatTok{4.59}\NormalTok{, }\FloatTok{14.96}\NormalTok{,  }\FloatTok{9.68}\NormalTok{, }\FloatTok{13.54}\NormalTok{,  }\FloatTok{6.42}\NormalTok{, }\FloatTok{11.03}\NormalTok{,  }
       \FloatTok{3.53}\NormalTok{, }\FloatTok{16.91}\NormalTok{,  }\FloatTok{9.52}\NormalTok{,  }\FloatTok{8.16}\NormalTok{,  }\FloatTok{8.97}\NormalTok{,  }\FloatTok{8.32}\NormalTok{,  }\FloatTok{3.58}\NormalTok{, }\FloatTok{13.57}\NormalTok{,  }\FloatTok{9.99}\NormalTok{)}
\NormalTok{y <-}\StringTok{ }\KeywordTok{c}\NormalTok{(}\FloatTok{36.69}\NormalTok{, }\FloatTok{6.39}\NormalTok{, }\FloatTok{49.59}\NormalTok{, }\FloatTok{45.65}\NormalTok{, }\FloatTok{52.18}\NormalTok{, }\FloatTok{27.66}\NormalTok{, }\FloatTok{79.35}\NormalTok{, }\FloatTok{54.10}\NormalTok{, }\FloatTok{71.01}\NormalTok{, }\FloatTok{34.60}\NormalTok{, }\FloatTok{61.17}\NormalTok{, }
       \FloatTok{22.79}\NormalTok{, }\FloatTok{91.20}\NormalTok{, }\FloatTok{50.57}\NormalTok{, }\FloatTok{44.11}\NormalTok{, }\FloatTok{53.51}\NormalTok{, }\FloatTok{45.96}\NormalTok{, }\FloatTok{22.20}\NormalTok{, }\FloatTok{73.01}\NormalTok{, }\FloatTok{55.70}\NormalTok{)}
\end{Highlighting}
\end{Shaded}

\textbf{1.} (a) Create a scatter plot to visualize the data
(\emph{Hint}: you may want to start with making a data frame and then
use \texttt{geom\_point()}). Do you think there is a strong linear
association between \texttt{x} and \texttt{y}?

\begin{enumerate}
\def\labelenumi{(\alph{enumi})}
\setcounter{enumi}{1}
\tightlist
\item
  Compute the sample correlation coefficient between \texttt{x} and
  \texttt{y}. Is your result consisent with your answer in (a)?
\end{enumerate}

\bigskip

\textbf{2.} (a) Assume that \texttt{y} is an outcome in a certain
experiment, and \texttt{x} is a predictor. Find the best fitting line
describing the association between \texttt{x} and \texttt{y} by
specifying its \(y\)-intercept (\(\beta_0\)) and slope (\(\beta_1\)).

\begin{enumerate}
\def\labelenumi{(\alph{enumi})}
\setcounter{enumi}{1}
\tightlist
\item
  Suppose that a new \texttt{x} value came in, say 13. Estimate the
  corresponding \texttt{y} value using the best fitting line you
  obtained in part (a) above.
\end{enumerate}

\bigskip
\bigskip

(For questions \textbf{3, 4, 5}) Aerial surveys sometimes rely on visual
methods to estimate the number of animals in an area. For example, to
study snow geese in their summer range areas west of Hudson Bay in
Canada, a small aircraft was used to fly over the range, and when a
flock of geese was spotted, an experienced person estimated the number
of geese in the flock. To investigate the reliability of this method of
counting, an experiment was conducted in which an airplane carrying two
observers flew over \(n=45\) flocks, and each observer made an
independent estimate of the number of birds in each flock. Also, a
photograph of the flock was taken so that a more or less exact count of
the number of birds in the flock could be made. The resulting data are
given in the attached data file \texttt{snowgeese.tsv}, which can be
downloaded from
\href{https://junglee0713.netlify.com/snowgeese.tsv}{\textbf{HERE}}. The
three variables in the data sets are \texttt{photo} = photo count;
\texttt{obs1} = aerial count by observer 1; and \texttt{obs2} = aerial
count by observer 2.

\textbf{3.} Read in the data set and draw the scatter plots of
\texttt{photo} (response) vs \texttt{obs1} (explanatory) and
\texttt{photo} (response) vs \texttt{obs2} (explanatory).

\bigskip

\textbf{4.} Compute the regression of \texttt{photo} on \texttt{obs1}
using \texttt{lm()} (That is, find the best fitting line). Overlay the
best fitting line to the scatter plot.

\bigskip

\textbf{5.} Suppose that observer 2 made a count of 55 in an additional
aerial survey performed later. What is your best guess for the actual
count?


\end{document}
